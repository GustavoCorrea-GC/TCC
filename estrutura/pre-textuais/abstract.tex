% ABSTRACT--------------------------------------------------------------------------------

\begin{resumo}[ABSTRACT]
\begin{SingleSpacing}

% Não altere esta seção do texto--------------------------------------------------------
\imprimirautorcitacao. \imprimirtitleabstract. \imprimirdata. \pageref {LastPage} f. \imprimirprojeto\ – \imprimirprograma, \imprimirinstituicao. \imprimirlocal, \imprimirdata.\\
%---------------------------------------------------------------------------------------

With the widespread use of the internet of things, the concept in which embedded devices are connected to the internet, there is a need to automatically update the firmware of these devices for corrections or enhancements. There are currently a wide variety of implementations of this functionality, but the lack of a standard makes it difficult to use widely.
Thus, this work offers an Over-The-Air firmware update solution for IoT devices, using as widespread libraries LwIP, FatFs and Mbed TLS.
The proposed system intends to provide an API, which can be integrated into any embedded platform, which will obtain the new firmware from a server, and a bootloader that performs the entire firmware change process.

\textbf{Keywords}: Update. Firmware. Over-The-Air. Portable. IoT.

\end{SingleSpacing}
\end{resumo}

% OBSERVAÇÕES---------------------------------------------------------------------------
% Altere o texto inserindo o Abstract do seu trabalho.
% Escolha de 3 a 5 palavras ou termos que descrevam bem o seu trabalho 
