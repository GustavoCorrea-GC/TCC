% RESUMO--------------------------------------------------------------------------------

\begin{resumo}[RESUMO]
\begin{SingleSpacing}

% Não altere esta seção do texto--------------------------------------------------------
\imprimirautorcitacao. \imprimirtitulo. \imprimirdata. \pageref {LastPage} f. \imprimirprojeto\ – \imprimirdepartamento, \imprimirinstituicao. \imprimirlocal, \imprimirdata.\\
%---------------------------------------------------------------------------------------

Com a necessidade de se criar um sistema de atualização de \textit{firmware} que não dependa de uma interação fisíca com o \textit{hardware}, e que seja portável para todas as plataformas de sistemas embarcados que utilizam comunicação via internet. 
Esse trabalho propõem uma solução \textit{open-source} de atualização de \textit{firmware Over-The-Air}, utilizando as bibliotecas amplamente difundidas LwIP, FatFs e Mbed TLS.
Este sistema se baseia em uma API que pode ser integrada a qualquer plataforma embarcada e um \textit{bootloader} que fazem todo o processo de obtenção e atualização de \textit{firmware}.

\textbf{Palavras-chave}: Atualização. Firmware. Over-The-Air. Portável. Sistemas Embarcados.

%O Resumo é um elemento obrigatório em tese, dissertação, monografia e TCC, constituído de uma seqüência de frases concisas e objetivas, fornecendo uma visão rápida e clara do conteúdo do estudo. O texto deverá conter no máximo 500 palavras e ser antecedido
%pela referência do estudo. Também, não deve conter citações. O resumo deve ser redigido em parágrafo único, espaçamento simples e seguido das palavras representativas do conteúdo do estudo, isto é, palavras-chave, em número de três a cinco, separadas entre si por ponto e finalizadas também por ponto. Usar o verbo na terceira pessoa do singular, com linguagem impessoal, bem como fazer uso, preferencialmente, da voz ativa. Texto contendo um único parágrafo.\\



\end{SingleSpacing}
\end{resumo}

% OBSERVAÇÕES---------------------------------------------------------------------------
% Altere o texto inserindo o Resumo do seu trabalho.
% Escolha de 3 a 5 palavras ou termos que descrevam bem o seu trabalho 

