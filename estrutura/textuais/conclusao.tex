% CONCLUSÃO--------------------------------------------------------------------

\chapter{CONCLUSÃO}
\label{chap:conclusao}
O trabalho conseguiu alcançar o objetivo especifico de criar um sistema de atualização de firmware Over-The-Air utilizando as bibliotecas FatFs, MBED TLS e LWIP. Todos os objetivos específicos também foram atingidos visto que foi desenvolvido um \bootloader\ que utiliza o sistema de aquivo FAT, foi implementada uma API que faz a comunicação com o servidor, armazena arquivos no cartão SD e também foi comprovado o funcionamento do sistema na plataforma embarcada STM32F746NGH6-DISCOVERY.

Após a conclusão do trabalho foi possível observar que o \bootloader\ desenvolvido é muito mais portável que inicialmente se planejava. Como o processo de troca de firmware foi desenvolvido utilizando o HAL da STM, foi possível fazer com que o \bootloader\ funcione para toda a linha de microcontroladores da familia STM32 que possuam a biblioteca FatFS e acesso a um cartão SD. Alem de ser um \bootloader\ robusto que consegue se recuperar em caso de falhas como quedas de energia durante o processo de troca de firmware.

A API que faz a busca dos arquivos no servidor HTTP e os armazena no cartão SD se mostrou perfeitamente funcional, conseguindo fazer a comparação de versões e a verificação de integridade do firmware baixado a partir de um hash obtido do servidor, deixando o sistema muito mais seguro contra falhas, além de ter a habilidade de conseguir encerrar todos os processos para que o \bootloader seja iniciado novamente.


A implementação do sistema de atualização implementado na plataforma embarcada STM32F746NGH6 se mostrou efetuada com sucesso, conseguindo mostrar como diferentes firmwares puderam ser trocados de forma rápida, eficiente e dando ao usuário sinalizações que mostravam cada etapa do sistema.

Um dos revezes deste trabalho foi o fato que o sistema de atualização ocupa um espaço consideravel na memória flash dos microcontroladores, fazendo com que o sistema não funcione em hardwares com pouca memória, assim diminuindo a usabilidade do sistema, mas ainda é um ferramenta muito util e poderosa para sistemas mais robustos. Fazendo que que se abra a possibilidade de trabalhos futuros que possam melhorar ainda mais o sistema criado.
\section{TRABALHOS FUTUROS}
\label{sec:trabalhosFuturos}

Com o intuito de se tornar ainda mais robusto e ocupar uma quantidade de memória menor, este capitulo aborda algumas propostas de melhoria do sistema que podem ser feitas em trabalhos futuros. Essa proposta faz com que o sistema seja mais atrativo para ser utilizado em sistemas com uma quantidade menor de memória.

Uma proposta consiste em utilizar todas as ferramentas já utilizadas no bootloader no próprio firmware em será atualizado. O firmware pode a partir de ponteiros de função utilizar as funções presentes na biblioteca do FatFS e do HAL que já estão implementadas no bootloader em sua aplicação, não sendo necessário reimplementa-las no firmware da aplicação, tornando o tamanho final do seu firmware da aplicação menor, assim fazendo com que placas de menor memória possam utilizar o sistema.

Pode-se utilizar outros meios de comunicação para a obtenção do novo firmware, fazendo com que não exista a necessidade de se utilizar toda a comunicação com o servidor e assim diminuindo a quantidade de memória ocupada pelo sistema. Uma implementação possivel pode ser o uso de UART para a obtenção do firmware, onde o firmware é dividido em diversos pacotes e esses são enviados aos poucos para o hardware e podem ser conferirdos CRCs no final de cada pacote para se obter uma prova de que o firmware obtido é integro.

