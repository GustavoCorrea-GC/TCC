% RESULTADOS-------------------------------------------------------------------

%\chapter{ANÁLISE E DISCUSSÃO DOS RESULTADOS}
\chapter{CRONOGRAMA}
%Cada capítulo deve conter uma pequena introdução (tipicamente, um ou dois parágrafos) que deve deixar claro o objetivo e o que será discutido no capítulo, bem como a organização do capítulo.
\begin{enumerate}
    \item Estudo bibliográfico; 
    \item Desenvolvimento da proposta;
    \item Familiarização com plataforma;
    \item Projeto de \textit{bootloader} e tarefas;
    \item Entrega do TCC1;
    \item Desenvolvimento do \textit{bootloader};
    \item Criação do servidor HTTP;
    \item Implementação das tarefas;
    \item Testes de integração;
    \item Escrita da monografia e artigo científico.
 \end{enumerate}
 
    \begin{table}[h]
    \scriptsize
        \begin{tabular}{|c|c|c|c|c|c|c|c|c|c|c|}
\hline
\textbf{Atividade} & \textbf{08/19} & \textbf{09/19} & \textbf{10/19} & \textbf{11/19} & \textbf{12/19} & \textbf{01/20} & \textbf{02/20} & \textbf{03/20} & \textbf{04/20} & \textbf{05/20} \\ \hline
\textbf{1}         & X              & X              & X              & X              & X              & X              & X              & X              & X              & X              \\ \hline
\textbf{2}         & X              &                &                &                &                &                &                &                &                &                \\ \hline
\textbf{3}         & X              & X              & X              & X              &                &                &                &                &                &                \\ \hline
\textbf{4}         &                & X              & X              & X              & X              &                &                &                &                &                \\ \hline
\textbf{5}         &                &                &                &                & X              &                &                &                &                &                \\ \hline
\textbf{6}         &                &                &                &                & X              & X              & X              &                &                &                \\ \hline
\textbf{7}         &                &                &                &                &                & X              &                &                &                &                \\ \hline
\textbf{8}         &                &                &                &                &                & X              & X              & X              & X              &                \\ \hline
\textbf{9}         &                &                &                &                &                &                &                & X              & X              & X              \\ \hline
\textbf{10}        & X              & X              & X              & X              & X              & X              & X              & X              & X              & X              \\ \hline
\end{tabular}
\end{table}

