% RESULTADOS-------------------------------------------------------------------
%Cada capítulo deve conter uma pequena introdução (tipicamente, um ou dois parágrafos) que deve deixar claro o objetivo e o que será discutido no capítulo, bem como a organização do capítulo.
%\chapter{ANÁLISE E DISCUSSÃO DOS RESULTADOS}
\chapter{CRONOGRAMA}
\label{Chap:Cronograma}

Para o desenvolvimento desta proposta foi definido o cronograma da \autoref{TabelaCronograma}

\begin{table}[H]
    \caption {Cronograma para o desenvolvimento do projeto}
    \label{TabelaCronograma}
    \scriptsize
    
        \begin{tabular}{ccccccccccc}
            
        \hline
        Atividades                                                             & Ago & Set & Out & Nov & Dez & Jan & Fev & Mar & Abr & Mai \\ \hline
        Estudo bibliográfico                                                   & X   & X   & X   & X   & X   &     & X   & X   & X   & X   \\ \hline
        Desenvolvimento da proposta                                            & X   &     &     &     &     &     &     &     &     &     \\ \hline
        Familiarização com plataforma                                          & X   & X   & X   & X   &     &     &     &     &     &     \\ \hline
        \begin{tabular}[c]{@{}c@{}}Projeto do \bootloader e API\end{tabular} &     & X   & X   & X   & X   &     &     &     &     &     \\ \hline
        Entrega do TCC1                                                        &     &     &     &     & X   &     &     &     &     &     \\ \hline
        Desenvolvimento do \bootloader                                          &     &     &     &     & X   &     & X   & X   &     &     \\ \hline
        Implementação da API                                                   &     &     &     &     &     &     & X   & X   & X   &     \\ \hline
        Testes de integração                                                   &     &     &     &     &     &     &     & X   & X   & X   \\ \hline
        Escrita da monografia e artigo científico                              & X   & X   & X   & X   & X   &     & X   & X   & X   & X   \\ \hline
        \end{tabular}
        \end{table}

        