% METODOLOGIA------------------------------------------------------------------

\chapter{SISTEMA OPEN-SOURCE PARA A ATUALIZAÇÃO DE FIRMWARE OVER-THE-AIR BASEADO NAS BIBLIOTECAS LWIP, MBED TLS E FATFS}
\label{chap:metodologia}
Neste capitulo é retratado como será o funcionamento do sistema que será desenvolvido, mostrada uma visão geral do software, listada as funcionalidades de cada uma das partes do software, explicada sua atividade e sua implementação, os problemas enfrentados durante o processo de desenvolvimento e as ferramentas utilizadas para o seu teste, Possiveis trabalhos futuros?. 

\section{VISÃO GERAL}
O software que será desenvolvido neste trabalho é dividido em duas partes, uma contendo o bootloader, que com o auxilio da biblioteca FatFs, que é um modulo generico do sistema de arquivo FAT, realizará a comunicação com o cartão SD, que conterá o novo firmware previamente recebido, e assim poderá substituir o software anterior da aplicação por um novo. Essa parte do software ficará armazenada em uma região da memoria que não poderá ser reescrita, então é uma peça do programa que não poderá ser substitúida. Será uma parte que não é portavel para todas as plataformas, ficando a cargo do projetista fazer o port para outras placas.

A outra parte deste trabalho será uma API contendo as demais funções necessárias para a comunicação com o servidor que enviará para o sistema o novo firmware, por meio do uso da biblioteca LwIP, garantirá a segurança dessa comunicação com a utilização de uma camada extra de segurança com biblioteca Mbed TLS. Essa API irá conectar-se a um servidor em um intervalo de tempo determinado pelo projetista, para consulta irá verificar a disponibilidade de uma nova versão de software. 

Após a confirmação de um novo firmware ser confirmada, em um tempo determinado pelo projetista a api irá se comunicar novamente com o servido com o intuito de fazer o download desta nova versão, e a armazenar em um endereço de memória predeterminado no cartão SD do sistema, para que então o bootloader entre em ação após uma reinicialização.
A API será uma peça de software que poderá ser substituida e atualizada em conjunto com as demais aplicações do sistema, como, as bibliotecas LwIP, Mbed TSL, o sistema operacional, entre outras peças de software utilizadas pelo sistema. 

Por conter biliotecas já conhecidas e vastamente utilizadas por desenvolvedores de sistemas embarcados, para que assim projetos que necessitem fazer comunicação segura via rede, leitura e escrita de cartões SD, possam utilizar esse sistema de modo a poupar espaço na memória, visando a reutilização dessas bibliotecas, portanto, o sistema pode ser amplamente utilizado. Será utilizada o \textit{kit} de desenvolvimento STM32F7 \textit{Discovery}, onde será inicialmente desenvolvida a API e suas tarefas e o \textit{bootloader} que serão os principais componentes desse sistema.


%Com a  As tarefas serão produzidas a partir de b
%sistema possa conter intersecções com trechos de codigos utilizados em varios projetos que necessitam

%Adicionar diagrama de blocos!!!!

%A partir de um arquivo de \textit{linker}, a memória da plataforma será customizada a fim de abrigar os arquivos necessários para o sistema e protege-los de eventuais sobre-escritas que podem vir a ocorrer. Esse arquivo de \textit{linker}, assim como o \textit{bootloader}, será escrito somente para a plataforma STM32F7, visto que cada plataforma tem suas próprias características como, tamanho de memória e endereços diferentes para cada fabricante e/ou arquitetura. A seguir será explicado parcialmente como funcionarão as funções das tarefas, do \textit{bootloader}, e do servidor HTTP.
%Cada capítulo deve conter uma pequena introdução (tipicamente, um ou dois parágrafos) que deve deixar claro o objetivo e o que será discutido no capítulo, bem como a organização do capítulo.

\section{O \textit{BOOTLOADER}}
\label{sec:Bootloader}

O \textit{bootloader} será responsável em fazer a validação e troca de cada versão de \textit{firmware} instalado no sistema embarcado. Sempre que o sistema for iniciado, o \textit{bootloader} fará a procura de um novo \textit{firmware} na memória interna do sistema. Esse processo, irá verificar o \textit{hash} da nova versão, verificando a integridade e origem do \textit{software}, para então poder ocorrer a atualização. Caso haja alguma falha durante esse processo, o \textit{bootloader} terá a habilidade de verificar esse erro e corrigi-lo, revertendo a atualização, e instalando o \textit{firmware} anterior. O funcionamento do \textit{bootloader} pode ser observado na figura \ref{Diagrama Bootloader}.

\begin{figure}[H]
    \scriptsize
     \centering
     \includegraphics[scale=0.39]{dados/figuras/DiagramaBootloader.png}
     \caption{Diagrama de funcionamento do \textit{bootloader}. Fonte: autoria própria.}
     \label{Diagrama Bootloader}
\end{figure}
%Inserir seu texto aqui...

\section{SERVIDOR HTTP}
\label{sec:ServidorHTTP}

A partir de um computador conectado à mesma rede que o sistema embarcado, haverá um servidor HTTP que ficará responsável por esperar requisições do dispositivo, para consultar a disponibilidade de uma versão atualizada do \textit{software}, e após a confirmação dessa nova versão, esse servidor irá enviar o \textit{firmware} para a plataforma embarcada.
%Inserir seu texto aqui...

\section{TAREFAS DO SISTEMA}
\label{sec:Tarefassistema}

\subsection{COMUNICAÇÃO}

Com o uso da biblioteca LwIP e MbedTLS essa tarefa estará responsável por criar uma comunicação segura entre o \textit{hardware} e o servidor HTTP, a partir dessa comunicação será feito a verificação do sinal de disponibilidade de novo \textit{software} e \textit{download} do mesmo quando o sistema estiver ocioso. 

\subsection{ARMAZENAMENTO}

A partir da utilização da biblioteca FatFs, essa tarefa fará a leitura e escrita sobre um cartão SD instalado no \textit{hardware}. Quando houver a transferência de uma nova versão, tanto o programa anterior quanto o novo, serão colocadas nesta memória, para futuras instalações que serão realizadas pelo \textit{bootloader}. A escolha da mídia de armazenamento de versões do \textit{software} estará a cargo do projetista, sendo escolhido para esse trabalho um cartão SD.


